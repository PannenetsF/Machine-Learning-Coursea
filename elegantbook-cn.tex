\documentclass[cn,11pt,chinese,black,simple]{elegantbook}

\def\mainclass{main}

\title{数字电路高层次综合设计}
% \subtitle{数字设计初步}

\author{Pannenets.F}
% \institute{微电子学院}
\date{\today}
% \version{4}
\bioinfo{分类}{笔记}

\extrainfo{Je reviendrai et je serai des millions. —— <<Spartacus>>}
\setcounter{tocdepth}{3}

\lstset{
  mathescape = false}
% \logo{logo-blue.png}
\cover{logo.jpg}

\input{needed.tex}

\begin{document}

\maketitle
\frontmatter

\chapter*{特别声明}
\markboth{Introduction}{前言}

本项目来源于 \href{https://github.com/ElegantLaTeX}{Elegant \LaTeX{}} 中的 \href{https://github.com/ElegantLaTeX/ElegantBook}{ElegantBook} ,进行了一些改动方便个人的使用。使用 \LaTeX{} 记笔记的初衷是\sout{自己的字太烂了}一份整洁的笔记可以取悦自己,\uline{并且使用 Markdown 或者 ADoc 等标记语言转码后会有一些不尽如人意的地方,不如在一开始就写成自己想要的模样。}

除去笔记之外,可能之后会开始写博客,把 Markdown 转码后夹在一个 Tex 里面,权当是对自己记录自己的一种仪式感。

\begin{center}
  把自己的碎碎念积攒成一本书,似乎也是一种很浪漫的事情。
\end{center}

\vskip 1.5cm


\begin{flushright}
Pannenets F
\today
\end{flushright}

\tableofcontents
%\listofchanges

\mainmatter

\part{part}

\chapter{chap}

\section{sec}





\end{document}
