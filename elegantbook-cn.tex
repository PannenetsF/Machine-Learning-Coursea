\documentclass[cn,11pt,chinese,black,simple]{elegantbook}

\def\mainclass{main}

\title{数字电路高层次综合设计}
% \subtitle{数字设计初步}

\author{Pannenets.F}
% \institute{微电子学院}
\date{\today}
% \version{4}
\bioinfo{分类}{笔记}

\extrainfo{Je reviendrai et je serai des millions. —— <<Spartacus>>}
\setcounter{tocdepth}{3}

\lstset{
  mathescape = false}
% \logo{logo-blue.png}
\cover{logo.jpg}

% 微分号
\newcommand{\dd}[1]{\mathrm{d}#1}
\newcommand{\pp}[1]{\partial{}#1}

% FT LT ZT
\newcommand{\ft}[1]{\mathscr{F}[#1]}
\newcommand{\fta}{\xrightarrow{\mathscr{F}}}
\newcommand{\lt}[1]{\mathscr{L}[#1]}
\newcommand{\lta}{\xrightarrow{\mathscr{L}}}
\newcommand{\zt}[1]{\mathscr{Z}[#1]}
\newcommand{\zta}{\xrightarrow{\mathscr{Z}}}

% 积分求和号

\newcommand{\dsum}{\displaystyle\sum}
\newcommand{\aint}{\int_{-\infty}^{+\infty}}

% 简易图片插入
\newcommand{\qfig}[3][nolabel]{
  \begin{figure}[!htb]
      \centering
      \includegraphics[width=0.6\textwidth]{#2}
      \caption{#3}
      \label{\chapname :#1}
  \end{figure}
}

% 表格
\renewcommand\arraystretch{1.5}

% 日期
% \usepackage{ctex}
\usepackage[numbered,framed]{matlab-prettifier}

\lstset{
  language = octave,
  style = Matlab-editor,
  basicstyle = \mlttfamily,
  escapechar = ",
  mlshowsectionrules = true,
}
% 
% \newenvironment{ocode}{\begin{lstlisting}[language=octave]}{\end{lstlisting}}

\lstnewenvironment{ocode}[1][]{%
    \lstset{language=octave,#1}}{}%


\begin{document}

\maketitle
\frontmatter

\chapter*{特别声明}
\markboth{Introduction}{前言}

本项目来源于 \href{https://github.com/ElegantLaTeX}{Elegant \LaTeX{}} 中的 \href{https://github.com/ElegantLaTeX/ElegantBook}{ElegantBook} ,进行了一些改动方便个人的使用。使用 \LaTeX{} 记笔记的初衷是\sout{自己的字太烂了}一份整洁的笔记可以取悦自己,\uline{并且使用 Markdown 或者 ADoc 等标记语言转码后会有一些不尽如人意的地方,不如在一开始就写成自己想要的模样。}

除去笔记之外,可能之后会开始写博客,把 Markdown 转码后夹在一个 Tex 里面,权当是对自己记录自己的一种仪式感。

\begin{center}
  把自己的碎碎念积攒成一本书,似乎也是一种很浪漫的事情。
\end{center}

\vskip 1.5cm


\begin{flushright}
Pannenets F
\today
\end{flushright}

\tableofcontents
%\listofchanges

\mainmatter

\part{part}

\chapter{chap}

\section{sec}





\end{document}
