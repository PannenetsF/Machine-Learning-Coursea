\ifx\mainclass\undefined
\documentclass[black,simple]{../elegantbook}
% \documentclass{book}
% 微分号
\newcommand{\dd}[1]{\mathrm{d}#1}
\newcommand{\pp}[1]{\partial{}#1}

% FT LT ZT
\newcommand{\ft}[1]{\mathscr{F}[#1]}
\newcommand{\fta}{\xrightarrow{\mathscr{F}}}
\newcommand{\lt}[1]{\mathscr{L}[#1]}
\newcommand{\lta}{\xrightarrow{\mathscr{L}}}
\newcommand{\zt}[1]{\mathscr{Z}[#1]}
\newcommand{\zta}{\xrightarrow{\mathscr{Z}}}

% 积分求和号

\newcommand{\dsum}{\displaystyle\sum}
\newcommand{\aint}{\int_{-\infty}^{+\infty}}

% 简易图片插入
\newcommand{\qfig}[3][nolabel]{
  \begin{figure}[!htb]
      \centering
      \includegraphics[width=0.6\textwidth]{#2}
      \caption{#3}
      \label{\chapname :#1}
  \end{figure}
}

% 表格
\renewcommand\arraystretch{1.5}

% 日期
% \usepackage{ctex}
\usepackage[numbered,framed]{matlab-prettifier}

\lstset{
  language = octave,
  style = Matlab-editor,
  basicstyle = \mlttfamily,
  escapechar = ",
  mlshowsectionrules = true,
}
% 
% \newenvironment{ocode}{\begin{lstlisting}[language=octave]}{\end{lstlisting}}

\lstnewenvironment{ocode}[1][]{%
    \lstset{language=octave,#1}}{}%

\begin{document}
\fi 
\def\chapname{03classification}

% Start Here
\chapter{Classification}

\section{Classification and Representation}

\subsection{Classification}

Classification is to predict the varible \(y\) into discrete values, in other workds, an assignment to  classify features into classes. For example, divide eamils by spam or not. 

We can set a list of threshold of some parameters of hypothesis to devide the dataset into different classes.

But if we apply linear regression, some extreme data will have a nonnegligible effect on the hypothesis.

As we all know, the hypothesis usually gives a continuous value,  while the classification problem gives discrete values. But apply hypothesis and map values into discrete values sometimes cannot work well.

Here we will focus on the binary classification which has a postive class and a negetive class.

\subsection{Hypothesis Representation}

A logistic regression model want it's \(h_\theta (x) \in [0, 1]\). We could apply \textbf{Sigmoid Function}:

\[
g(z) = \frac{1}{1 + e^{-z}}    
\]
\

It's like \qfig[]{c3p1.png}{Sigmoid Function}

Then, we get: 

\[
h_\theta (x) = g(\theta^T x)    
\]

Let's interprete the hypothesis' output: \(h_\theta(x)\) is the estimated probability that \(y = 1\) on input \(x\). That is \textbf{probability that \(y = 1\), given a \(x\) parameterized by \(\theta\)}:

\[
h_\theta (x) = P(y = 1 | x;\theta)    
\]

\subsection{Decision Boundary}

Why Sigmoid Function makes sense? When \(h_\theta (x) > 0.5\) we predict \(y = 1\) and predict \(y = 0\) in else condition. And it means that \(\theta^T x > 0\) or not. 

\[
h_\theta (x) = g(\theta^T x)    
\]

By Sigmoid Function we can classify the hypothesis by it's values.

When we get 2 varibles, like \(h_\theta(x) = g(\theta_0 + \theta_1 x_1 + \theta_2 x_2)\) and we still make binary classification, wo we can still predict \(y = 1\) if \(h_\theta(x) \ge 0\) and \(y = 0\) if \(h_\theta(x) \le 0\).

Further more, how about we need a non-linear decision boundaries, like a circle or a a ellipse? Just use non-linear functions! For a ellipse:

\[
h_\theta(x) = g(\theta_0 + \theta_1 x_1 + \theta_2 x_2 + \theta_3 x_1^2 + \theta_4 x_2^2)     
\]

And we define the boundary like: 

\[
\left\{
\begin{aligned}
    y &= 1, \text{ if } x_1^2 + x_2^2 \ge 1\\
    y &= 0, \text{ else}
\end{aligned}    
\right.
\]

\section{Logistic Regression Model}

\subsection{Cost Function}

For a classification problem, we need the cost function, too. This is related to how to choose the parameters.

In chapters before, we defined as a sum of squared error:
 
\[
\text{Cost} \left(h_\theta(x^{(i)}, y^{(i)})\right) = \frac{1}{2} \left(h_\theta(x^{(i)} )- y^{(i)}\right) ^2
\]

\[
J(\theta) = \frac{1}{m} \sum_{i = 1}^m \text{Cost} (h_\theta(x^{(i)}, y^{(i)}))
\]

For a simple cost function, it's non-convex so it does not guarantee to converge.

In logistic regression cost function, we define:

\[
\text{Cost}\left(h_\theta(x), y\right) = \left\{
\begin{aligned}
    - \log (h_\theta(x)) & \text{ if } y = 1\\
    - \log (1 - h_\theta(x)) & \text{ if } y = 0
\end{aligned}    
\right.
\]

If the hypothesis gives an approximate output as \(y\), the cost is rather small; but when it goes to the other end, the cost will grow in a very fast speed.

And this cost function could be convex in our problem.

\subsection{Simplified Cost Function and Gradient Descent}

Let's re-write the cost function in a more compact way as: 

\[\text{Cost}\left(h_\theta(x), y\right) = -y \log (h_\theta(x), y) - (1-y) \log (1-h_\theta(x))\]

\[
\begin{aligned}
    J(\theta) &= \frac{1}{m} \sum_{i = 1}^m \text{Cost} (h_\theta(x^{(i)}, y^{(i)}) \\
    &=  -\frac{1}{m}\sum_{i=1}^m\left[y^{(i)} \log (h_\theta(x^{(i)})) + (1-y^{(i)}) \log (1-h_\theta(x^{(i)}))\right]
\end{aligned}    
\]

To fit parameters \(\theta\), we need to minimize the \(J(\theta)\). Then we can apply the gradient descent as in previous chapters.


\[\begin{aligned}
    \theta_j := \theta_j - \alpha \frac{1}{m} \sum_{i = 0}^m (h_\theta(x^{(i)}) - y^{(i)})x_j^{(i)}, \text{where } \theta = 0, 1, 2, \cdots , n
\end{aligned}\]

\textbf{Note}: the \(h_\theta(x)\) is different! Now we have \(h_\theta(x) = 1 / (1 + e^{-\theta x})\).

\subsection{Advanced Optimization}

With this section, we can get logistic regression run more quickly. Like, gradient descent, conjugate gradient, BFGS\footnote{Broyden-Fletcher-Goldfarb-Shanno algorithm, }, L-BFGS\footnote{Limited-memory BFGS}

We can just simply call \lstinline{fminunc} to get gradient descent.


\section{Multi-class Classification}


For multiclass problem, we can encode each class with some num like 1, 2, 3, 4 and so on.
One-verse-all algorithm is to seperate different class to a special postive class in turn. Or: train a logistic regersss     classifier \(h_\theta^{(i)}(x)\) for each class \(i\) to predict the probability that \(y = i\). Finally, pick the \(\max_i h_\theta^{(i)}(x)\).

\section{Solve Overfitting: Regularization}

\subsection{The Problem of Overfitting}

What's overfitting? If a little bit more complicate dataset cannot be fitted with a line, the linear fit will have a high preconception or bias(underfitting). If we apply a polynomial regression with too high order, hte learned hypothesis will fit the dataset very well but fall to have the ablity to generalize the problem(overfitting).

If we want to address the overfitting, we can

\begin{itemize}
    \item reduce number of features
    \begin{itemize}
        \item manually select features to keep
        \item model selecttion algorithm
    \end{itemize}
    \item aplly regularization
    \begin{itemize}
        \item keep all features, but reduce magnitude or value of some parameters \(\theta_j\)
        \item works well when get lots of features
    \end{itemize}
\end{itemize}

\subsection{Cost Function for Regularization}

If we want to make some parameters really small, we can add some terms like:

\[J(\theta) = \frac{1}{2m}\sum_{i=1}^m(h_\theta(x^{(i)}) - y^{(i)})^2
+ 1000 \theta_3^2 + 1000 \theta_4^2
\]

Small parameters will turn to simpler hypothesis and get more smoth and less prone to overfit. If we want to shrink all parameters, the cost could be implemented as\footnote{remeber that: \(\theta\) starts with 0, but we just ignore the first term }:

\[J(\theta) = \frac{1}{2m}\left[\sum_{i=1}^m(h_\theta(x^{(i)}) - y^{(i)})^2 + \lambda \sum_{i=1}^n\theta_i^2 \right]\]

The latter term is called regularization term. The regularization will put off a more general hypothesis.

\subsection{Regularized Linear Regression}

If we do gradient descent, the gradient will be different for \(\theta_0\) because the extra term does nothing with it.


\[\theta_j := \left\{
    \begin{aligned}
     \theta_j - \alpha \frac{1}{m} \left[ \sum_{i = 0}^m (h_\theta(x^{(i)}) - y^{(i)})x_j^{(i)}+ \frac{\lambda}{m}\theta_j\right], &\text{ where } \theta = 1, 2, \cdots , n\\
     \theta_j - \alpha \frac{1}{m} \sum_{i = 0}^m (h_\theta(x^{(i)}) - y^{(i)})x_j^{(i)}, &\text{ where } \theta = 0
\end{aligned}\right.\]

Similarly, we can still apply the normal equation in conditions: 

\[
    \theta = \left(X^T X + \lambda M \right)^{-1} X^T y, \text{ where } M = \left[\begin{matrix}
        0 & & & & \\
        & 1 & & & \\
        & & 1 & & \\
        & & & \ddots & \\
        & & & & 1
    \end{matrix}\right]
\]


\subsection{Regularized Logistic Regression}



\[
    J(\theta)=  -\frac{1}{m}\left[y^{(i)} \log (h_\theta(x^{(i)}), y^{(i)}) + (1-y^{(i)}) \log (1-h_\theta(x^{(i)}))\right] + \frac{\lambda}{2 m} \theta_i^2
\]

\[
    \theta_j := \left\{
    \begin{aligned}
        \theta_j - \alpha \frac{1}{m} \sum_{i = 0}^m (h_\theta(x^{(i)}) - y^{(i)})x_j^{(i)}, &\text{ where } \theta = 1, 2, \cdots , n\\
        \theta_j - \alpha \left[\frac{1}{m} \sum_{i = 0}^m (h_\theta(x^{(i)}) - y^{(i)})x_j^{(i)} + \frac{\lambda}{m}\theta_j\right], &\text{ where } \theta = 0
    \end{aligned}    
    \right.
\]


\section*{Words}

ameliorate 改善

preconception 偏见

contort 曲解

penalize 惩罚

prone 倾向

% End Here




\let\chapname\undefined
\ifx\mainclass\undefined
\end{document}
\fi 