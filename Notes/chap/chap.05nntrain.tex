\ifx\mainclass\undefined
\documentclass[en,11pt,english,black,simple]{../elegantbook}
% 微分号
\newcommand{\dd}[1]{\mathrm{d}#1}
\newcommand{\pp}[1]{\partial{}#1}

% FT LT ZT
\newcommand{\ft}[1]{\mathscr{F}[#1]}
\newcommand{\fta}{\xrightarrow{\mathscr{F}}}
\newcommand{\lt}[1]{\mathscr{L}[#1]}
\newcommand{\lta}{\xrightarrow{\mathscr{L}}}
\newcommand{\zt}[1]{\mathscr{Z}[#1]}
\newcommand{\zta}{\xrightarrow{\mathscr{Z}}}

% 积分求和号

\newcommand{\dsum}{\displaystyle\sum}
\newcommand{\aint}{\int_{-\infty}^{+\infty}}

% 简易图片插入
\newcommand{\qfig}[3][nolabel]{
  \begin{figure}[!htb]
      \centering
      \includegraphics[width=0.6\textwidth]{#2}
      \caption{#3}
      \label{\chapname :#1}
  \end{figure}
}

% 表格
\renewcommand\arraystretch{1.5}

% 日期
% \usepackage{ctex}
\usepackage[numbered,framed]{matlab-prettifier}

\lstset{
  language = octave,
  style = Matlab-editor,
  basicstyle = \mlttfamily,
  escapechar = ",
  mlshowsectionrules = true,
}
% 
% \newenvironment{ocode}{\begin{lstlisting}[language=octave]}{\end{lstlisting}}

\lstnewenvironment{ocode}[1][]{%
    \lstset{language=octave,#1}}{}%

\begin{document}
\fi 
\def\chapname{05nntrain}

% Start Here
\chapter{Neural Network: Learning}

\section{Cost Function and Backpropagation}

\subsection{Cost Function}

Let's define symbols for n-class classification:

\begin{itemize}
    \item the input feature and its class: \({(x^{(1)}), y^{(1)}},{(x^{(2)}), y^{(2)}},\cdots,{(x^{(n)}), y^{(n)}}\)
    \item \(L\) is total number of layers
    \item \(s_l\) is the number of units in layer \(l\)
\end{itemize}

For logistic regression, the cost function is 


\[J(\theta) = \frac{1}{2m}\left[\sum_{i=1}^m(h_\theta(x^{(i)}) - y^{(i)})^2 + \lambda \sum_{i=1}^n\theta_i^2 \right]\]

For a neural network, it's:

\[\begin{aligned}
    J(\Theta)&=-\frac{1}{m}\left[\sum_{i=1}^{m} \sum_{k=1}^{K} y_{k}^{(i)} \log \left(h_{\Theta}\left(x^{(i)}\right)\right)_{k}+\left(1-y_{k}^{(i)}\right) \log \left(1-\left(h_{\Theta}\left(x^{(i)}\right)\right)_{k}\right)\right] \\
    &+\frac{\lambda}{2 m} \sum_{l=1}^{L-1} \sum_{i=1}^{s_{l}} \sum_{j=1}^{s_{l+1}}\left(\Theta_{j i}^{(l)}\right)^{2}\\
    \text{where: }& \quad \quad h_{\Theta}(x) \in \mathbb{R}^{K}, \quad\left(h_{\Theta}(x)\right)_{i}=i^{t h} \text { output } \\
\end{aligned}\]

The first term is for all \(K\) dimension output, and the last is the regular term of all weight in the neural network.

\subsection{Backpropagation Algorithm}

Backpropagation algorithm is a way to minimize the cost.

To use gradient descent, we need \(J(\theta)\)
 and \(\dfrac{\pp{J(\theta)}}{\pp{\Theta_{i,j}^{(l)}}}\).

So we have to compute the partical terms. We define the error of the \(L\) layer's node \(j\):

\[\delta_j^{(l)} = a^{(l)}_j - y_j\]

Then, for earlier layers:


\[\delta^{(l-1)} = (\Theta^{(l-1)})^T \delta^{(l)} .* g'(z^{(l-1)})\]

Where: 

\[g'(z^{(l)}) = a^{(l)} .* (1-a^{(l)})\]

Finally:

\[\frac{\pp{}}{\pp{\Theta_{i,j}^{(l)}}} J(\Theta) = a_j^{(l)} \delta_i^{(l+1)}, \text{when } \lambda = 0\]

For a training set  \({(x^{(1)}), y^{(1)}},{(x^{(2)}), y^{(2)}},\cdots,{(x^{(m)}), y^{(m)}}\):

\begin{lstlisting}
    Delta(l)(i,j) = 0
    for i = 1 : m
        set a(1) = x(i)
        compute for a(l) for l = 2,3,..,L
        with y(i), compute delta(L)
        then compute delta(L-1), ...,delta(1)
        Delta(l)(i,j) = Delta(l)(i,j) + a(l)(j) * delta(l+1)(i)
    endfor
\end{lstlisting}

Or in vector form: 

\[\Delta^{(l)} := \Delta^{(l)} + \delta^{(l+1)} (a^{(l)})^T\]



Then:

\[
\begin{aligned}
    D_{i,j}^{(l)} &= \frac{1}{m} \Delta_{i,j}^{(l)} + \lambda \Theta_{i,j}^{(i)}, \text{ if } j \neq 0\\
    D_{i,j}^{(l)} &= \frac{1}{m} \Delta_{i,j}^{(l)} + \lambda \Theta_{i,j}^{(i)}, \text{ if } j = 0
\end{aligned}
\]

And: 

\[\\
\dfrac{\pp{}}{\pp{\Theta_{i,j}^{(l)}}}J(\Theta) 1= D_{i,j} ^{(l)}\]

\subsection{Backpropagation Intuition}

Backpropagation is more likely to be a blankbox than previous algorithm.  

What do forward propagation do in the NNs? Doing non-linear matrix multiplication by introducing bias. And similarly, the former error is due to latter layers. 

\section{Backpropagation in Practice}

\section{Application of Neural Network}


% End Here

\let\chapname\undefined
\ifx\mainclass\undefined
\end{document}
\fi 