\ifx\mainclass\undefined
\documentclass[en,11pt,english,black,simple]{../elegantbook}
% 微分号
\newcommand{\dd}[1]{\mathrm{d}#1}
\newcommand{\pp}[1]{\partial{}#1}

% FT LT ZT
\newcommand{\ft}[1]{\mathscr{F}[#1]}
\newcommand{\fta}{\xrightarrow{\mathscr{F}}}
\newcommand{\lt}[1]{\mathscr{L}[#1]}
\newcommand{\lta}{\xrightarrow{\mathscr{L}}}
\newcommand{\zt}[1]{\mathscr{Z}[#1]}
\newcommand{\zta}{\xrightarrow{\mathscr{Z}}}

% 积分求和号

\newcommand{\dsum}{\displaystyle\sum}
\newcommand{\aint}{\int_{-\infty}^{+\infty}}

% 简易图片插入
\newcommand{\qfig}[3][nolabel]{
  \begin{figure}[!htb]
      \centering
      \includegraphics[width=0.6\textwidth]{#2}
      \caption{#3}
      \label{\chapname :#1}
  \end{figure}
}

% 表格
\renewcommand\arraystretch{1.5}

% 日期
% \usepackage{ctex}
\usepackage[numbered,framed]{matlab-prettifier}

\lstset{
  language = octave,
  style = Matlab-editor,
  basicstyle = \mlttfamily,
  escapechar = ",
  mlshowsectionrules = true,
}
% 
% \newenvironment{ocode}{\begin{lstlisting}[language=octave]}{\end{lstlisting}}

\lstnewenvironment{ocode}[1][]{%
    \lstset{language=octave,#1}}{}%

\begin{document}
\fi 
\def\chapname{09AnomalyDetection}

% Start Here
\chapter{Anomaly Detection}


\section{Motivation}

To tell something is working anomalously. The center is \(p(x) < \epsilon\) .

For example, it could be applied to fraud detection. 

\section{Gaussian Distribution }

If \(x\) is a distributed Guassian with mean \(\mu\) and variance \(\sigma^2\), it's \(x \sim N(\mu, \sigma^2) \)

With Guassian Distribution, we can estimate a dataset: 

\[\begin{aligned}
    \mu &= \frac{1}{m} \sum_{i=1}^m x^{(i)} \\
    \sigma^2 &= \frac{1}{m} \sum_{i=1}^m (x^{(i)}-\mu)^2 
\end{aligned}\]

\section{Density Estimation} 

Every componets will be a normal distribution. 

\section{Developing and Evaluating an Anomaly Detection System}

If we have some labeled data, part of them are anomalous and non-anomalous. Then we can compute whether an example is normal or not.


\section{Multivariate Gaussian Distribution}

Model the \(x\) rather than \(x_i\) separately. Then \(\mu \in \mathbb{R}\) and \(\Sigma \in \mathbb{R}^{n\times n}\) .

\[p(x,\mu,\Sigma) = \frac{1}{(2\pi)^{1/2}|\Sigma|^{1/2}} \exp(-\frac{1}{2}(x-\mu)^T \Sigma^{-1} (x-\mu))\] 

This could automatically captures correlations between features. (Need: \(m > 10 n\)) It's more expensive than original model.


% End Here

\let\chapname\undefined
\ifx\mainclass\undefined
\end{document}
\fi 

